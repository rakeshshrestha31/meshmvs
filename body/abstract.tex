\begin{abstract}
    Deep learning based 3D shape generation methods generally utilize latent features extracted from color images to encode the semantics of objects and guide the shape generation process.
    These color image semantics only implicitly encode 3D information, potentially limiting the accuracy of the generated shapes.
    In this paper we propose a multi-view mesh generation method which incorporates geometry information explicitly by using the features from intermediate depth representations of multi-view stereo and regularizing the 3D shapes against these depth images.
    First, our system predicts a coarse 3D volume from the color images by probabilistically merging voxel occupancy grids from the prediction of individual views.
    Then the depth images from multi-view stereo along with the rendered depth images of the coarse shape are used as a contrastive input whose features guide the refinement of the coarse shape through a series of graph convolution networks.
    Attention-based multi-view feature pooling is proposed to fuse the contrastive depth features from different viewpoints which are fed to the graph convolution networks.
    Notably, we achieve superior results than state-of-the-art multi-view shape generation methods with 34\% decrease in Chamfer distance to ground truth and 14\% increase in F1-score on ShapeNet dataset.

    % Additional constrains between the rendered depths of the predicted shapes and the predicted depth maps are introduced to further regularize the shape generation process.
    % A series of graph convolution networks are applied to refine the shape in a coarse to fine manner all of which utilize contrastive depth input at the current stage of refinement.

    % Our method extends recent single view shape generation method Mesh R-CNN~\cite{gkioxari2019meshrcnn} to multiple views and introduces contrastive feature extraction to get superior results than state-of-the-art multi-view shape generation method with 34\% decrease in Chamfer distance to ground truth and 14\% increase in F1-score on ShapeNet dataset.
\end{abstract}